\documentclass{article}

\usepackage[accepted]{icml2012}
\usepackage[utf8]{inputenc}
\usepackage{amsmath}
\usepackage{amssymb}
\usepackage{graphicx}
\usepackage{xspace}
\usepackage{natbib}
\usepackage{url}

\newcommand{\sklearn}{\textit{scikit-learn}\xspace}

% Dutch name sorting hack as per http://tex.stackexchange.com/a/40750/2806
\DeclareRobustCommand{\VAN}[3]{#2}

\begin{document}

\twocolumn[
\icmltitle{Scikit-Learn: a Machine Learning library in the Python ecosystem}

\icmlauthor{Gilles Louppe}{University of Liège, Belgium}
\icmlauthor{Gaël Varoquaux}{Parietal, INRIA Saclay, France}

\vskip 0.3in
]

\section*{Scikit-Learn}

The \sklearn\footnote{\url{http://scikit-
learn.org}}\footnote{\url{http://mloss.org/software/view/240}}
project~\citep{pedregosa2011} is an increasingly popular machine learning
library written in Python.  It is designed to be both  simple and efficient,
useful to both experts and non-experts, and reusable in a variety of contexts.
The  primary aim of the project is to provide a compendium of efficient
implementations of classic, well-established  machine learning algorithms.
Among other things, it includes implementations of classical supervised and
unsupervised learning algorithms, tools for model evaluation and selection, as
well as tools for preprocessing and feature engineering. \sklearn is
distributed under the 3-clause  BSD license, encouraging its free use in both
commercial and academic settings.

Started in 2007, \sklearn is developed by an international team of over a dozen
core developers, mostly researchers from various fields of science. The project
also benefits from many occasional contributors proposing small bugfixes or
improvements. Because of the large number of developers, emphasis is put on
keeping the project maintainable. Code must follow quality guidelines, such as
style consistency and unit-test coverage. Documentation and examples are
required for all features, and major changes must pass code review by at least
two developers not involved in the implementation of the proposed change.

All algorithms within \sklearn are offered through a simple and non-intrusive
API~\citep{buitinck2013api} consisting of a limited set of methods. Its
consistency across the package makes it very usable in practice: experimenting
with different learning algorithm is as simple as substituting a class
definition. Through composition interfaces, the library also offers powerful
mechanisms to express a wide variety of learning tasks within a small amount of
easy-to-read code. Finally, through duck-typing, the consistent API leads to a
library that is easily extensible, and allows user-defined estimators to be
incorporated into the \sklearn workflow without any explicit object
inheritance. % In particular, the power and extensibility of the \sklearn % API
% is evidenced by its large and growing user-base as well as the appearance % of
% third-party packages that follow our conventions.

\section*{Integration in the Python ecosystem}

The library has been designed to leverage standard tools of the scientific
Python ecosystem, including NumPy~\citep{vanderwalt2011} for efficient  storage
and manipulation of multi-dimensional arrays, SciPy~\citep{varoquaux2013scipy}
for more specialized data structures  (e.g. sparse matrices) and
implementations of lower-level scientific algorithms, and Matplotlib for
dynamic data visualization in two and three dimensions.

(todo)


\section*{Demonstrations}

This presentation will include demonstrations of some of the most powerful and
popular algorithms in \sklearn.

(todo)


{\small
\bibliographystyle{abbrvnat}
\DeclareRobustCommand{\VAN}[3]{#3}
\bibliography{sklearn-nips-mloss}
}

\end{document}

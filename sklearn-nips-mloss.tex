\documentclass{article}

\usepackage[accepted]{icml2012}
\usepackage[utf8]{inputenc}
\usepackage{amsmath}
\usepackage{amssymb}
\usepackage{graphicx}
\usepackage{natbib}

\begin{document}

\twocolumn[
\icmltitle{Scikit-Learn}

\icmlauthor{Gilles Louppe}{g.louppe@ulg.ac.be}
\icmladdress{University of Liège, Belgium}
% \icmlauthor{Gilles Louppe}{g.louppe@ulg.ac.be}
% \icmladdress{University of Liège, Belgium}

\vskip 0.3in
]

\section*{Machine Learning in Python}

Scikit-Learn~\citep{pedregosa2011} is an increasingly popular machine learning
library written in Python.  It is designed to be both  simple and efficient,
useful to both experts and non-experts, and reusable in a variety of contexts.
The  primary aim of the project is to provide a compendium of efficient
implementations of classic, well-established  machine learning algorithms with
a consistent and elegant API. Among other things, it includes implementations
of classical supervised and unsupervised learning algorithms, tools for model
evaluation and selection, as  well as tools for preprocessing and feature
engineering. Scikit-Learn is distributed under the 3-clause  BSD license,
encouraging its free use in both commercial and academic settings.


\section*{Integration}

Scikit-learn leverages the standard tools of the Scientific Python universe,
including NumPy for efficient  storage and manipulation of multi-dimensional
arrays, SciPy for more specialized data structures  (e.g. sparse matrices) and
implementations of lower-level scientific algorithms, and Matplotlib for
dynamic data visualization in two and three dimensions.


\section*{Software demonstration}

This presentation will include demonstrations of some of the most powerful and
popular algorithms in the  scikit-learn package.

\bibliographystyle{abbrvnat}
\bibliography{sklearn-nips-mloss}

\end{document}
